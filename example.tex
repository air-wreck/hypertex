\documentclass{article}
\usepackage{hypertex}

\begin{document}
\begin{html}
  <h1>Welcome to the Hyper\TeX{} Markup Language!</h1>
  <p>
    The Hyper\TeX{} Markup Language (HML) is a way to write \LaTeX{}
    using HTML syntax, by implementing a primitive HTML parser entirely
    in \LaTeX{} macros. For instance, here are some formatting examples:
    <ul>
      <li><em>Italicized text</em></li>
      <li><strong>Boldface text</strong></li>
      <li><s>Struck out text</s></li>
      <li><tt>Monospace text</tt></li>
    </ul>
    You can even <em><strong>combine styles!</strong></em> The
    possibilities are truly limited only by <s>your imagination</s> my
    extremely janky and incomplete HTML parser. It is also possible to
    inter-op with regular \LaTeX{} macros to some degree, so that you
    can achieve effects like \textsc{Small Caps} in the usual way.
  </p>
  <p>
    There is some amount of built-in error-checking. For example,
    \texttt{hypertex} is able to detect and recover from
    <unrecognized>unrecognized tags</unrecognized> and certain cases of
    <em>mismatched open/close tags</strong>.
  </p>
  <h2>Okay, but why would I want to do this?</h2>
  <p>
    For many decades, \LaTeX{} has reigned supreme as the document
    preparation markup language of choice for discerning professionals
    in technical fields. However, many new users complain that it's
    difficult to learn. That's where HML steps in: it lets you write
    directly in HTML, a markup language that is ``clean,'' ``free of
    weird stuff,'' and ``loved by millions around the world.''
  </p>
  <p>
    Okay yeah you wouldn't, I just thought it'd be funny.
  </p>
\end{html}
\end{document}
